\chapter{JSON}

JSON is used widely in TCS to represent data structures in configuration files, alert files, block files, and in inter-process communication. JSON has a simple and regular syntax that is fairly easy for humans to write, easy for computers to parse, and should be quite familiar to C, C++, and JavaScript programmers. 

\section{Dialect}

The dialect of JSON used in TCS is both extended and restricted compared to standard JSON.

It is extended by the addition of comment lines that can appear wherever white-space can appear in standard JSON. A comment line is a line that begins with zero or more space and horizontal tab characters followed by two slash characters (“//”). Formally, the slash character is U+002F SOLIDUS. Other types of comments (e.g., block comments introduced by “/*” and terminated by “*/”) are not allowed. Comment lines are treated as if they were empty lines. Our dialect of JSON can be converted to standard JSON by using a regular expression substitution to replace comment lines with empty lines.

It is restricted with respect to standard JSON in that it only uses array, object, and string values and does not use number, true, false, or null values. If a value represents a number, then the string representation should be used. This restriction ensures that value can be read and written without being changed.

The dialect of JSON is formally a very limited subset of JavaScript. Therefore, if you are editing data files manually, it might be useful to select an editing mode suitable for JavaScript.

\section{Encoding}

Data files should be encoded in UTF-8. ASCII is a subset of UTF-8, but ISO-8859-1 “Latin-1” is not.

\section{Values}

\subsection{Value Types}

Only array, object, and string values are used. Number and boolean values are represented by their string representation. For example, we do not write
\begin{quote}
\verb|0|\\
\verb|true|\\
\verb|false|\\
\end{quote}
but rather 
\begin{quote}
\verb|"0"|\\
\verb|"true"|\\
\verb|"false"|\\
\end{quote}
Null values are not used.

\subsection{Dates and Times}

Dates and time values are written as strings containing an IS0-8601 basic date or basic combined date and time.
The time zone is understood to be UTC; an explicit time zone must not be specified. The precision must not be specified more finely than seconds.

Examples of valid dates are:
\begin{quote}
\verb|"20101117T223815"| : 2010 Nov 17 22:38:15 UTC\\
\verb|"20101117T2238"  | : 2010 Nov 17 22:38:00 UTC\\
\verb|"20101117T22"    | : 2010 Nov 17 22:00:00 UTC\\
\verb|"20101117"       | : 2010 Nov 17 00:00:00 UTC
\end{quote}

\subsection{Angles}

A value representing an angle may be written as:

\begin{enumerate}
    \item a string containing a decimal number (and interpreted as \emph{radians} not \emph{degrees});
    \item a string containing sexagesimal notation with colons as separators and no white space;
    \item a strings containing a decimal number with a unit suffix (with no intermediate white space):
    \begin{enumerate}
        \item “r” for radians;
        \item “h” for hours;
        \item “m” for minutes;
        \item “s” for seconds;
        \item "ad" or “d” for degrees;
        \item “am” for arcminutes; and
        \item “as” for arcseconds.
    \end{enumerate}
\end{enumerate}

For hour angles and right ascensions, sexagesimal notation indicates hours, minutes, and seconds. For all other angles, including offsets, it indicates degrees, arcminutes, and arcseconds.

Examples of valid angles are:
\begin{quote}
\verb|"-22.5"    | : $-22.5$ \emph{radians}\\
\verb|"-22.5d"   | : $-22.5$ \emph{degrees}\\
\verb|"-01:30:00"| : $-1.5$ hours or $-1.5$ degrees, depending on the context\\
\verb|"0.5r"     | : $0.5$ radians\\
\verb|"-1.5h"    | : $-1.5$ hours\\
\verb|"5am"      | : $5$ arcminutes\\
\verb|"+20as"    | : $20$ arcseconds
\end{quote}

\subsection{Durations}

A value representing a duration may be written as:
\begin{enumerate}
\item a string containing a decimal number (and interpreted as decimal seconds);
\item a string containing sexagesimal notation with colons as separators and no white space;
\item a string containing a decimal number with a unit suffix (with no intermediate white space):
\begin{enumerate}
    \item “h” for hours;
    \item “m” for minutes; and
    \item “s” for seconds.
\end{enumerate}
\end{enumerate}

Sexagesimal notation indicates hours, minutes, and seconds.

Examples of valid durations are:
\begin{quote}
\verb|"60"      | :	60 seconds\\
\verb|"00:30:10”| : 30 minutes and 10 second\\
\verb|"60s"		| :	60 seconds\\
\verb|"10m"		| :	10 minutes\\
\verb|"1h"		| :	1 hour
\end{quote}

\subsection{Validation}

JSON written by hand is prone to errors such as missing commas. For this reason, it is often worthwhile checking JSON with one of the many on-line validators. The following validator is especially useful, as it tolerates comments:

\begin{quote}
\url{https://jsonformatter.curiousconcept.com/#}
\end{quote}

\section{Bibliography}

\begin{itemize}
    \item “Introducting JSON”, \url{https://www.json.org/json-en.html}
\end{itemize}