\chapter{Calibrations}

This chapter describes the routine calibrations procedures for {\projectname}.

\section{Twilight Flats}

The control system takes twilight flats each evening that conditions permit.

The flats are taken with the telescope pointed to $-3$ hours of HA and $+45$ degrees of declination and with the tracking turned off. 
The detector uses a read mode of
\ifcoatli
\verb|conventionaldefault| (which is \verb|1MHz-low|).
\fi
\ifddoti
\verb|default| (which is \verb|16MHz|).
\fi
The control system monitors the level in the flats, considers a flat to be good it its level is between 
\ifcoatli
3500 and 16000 DN, 
\fi
\ifddoti
1000 and 3000 DN,
\fi
and moves to the next filter either after acquiring a certain number of good flats (typically 7) or when the level is too low for the flat to be considered good.

\ifcoatli
HUITZI has many filters and the flats for each filter is associated with a different visit identifier, according to Table~\ref{table:twilight-flats-visits}.


\begin{table}
\begin{center}
\caption{Visit Identifiers for Twilight Flats}
\label{table:twilight-flats-visits}
\medskip
\begin{tabular}{ll}
\hline
Filter&Visit Identifier\\
\hline
$g$&0\\
$r$&1\\
$i$&2\\
$z$&3\\
$y$&4\\
$w$&5\\
$B$&10\\
$V$&11\\
$R$&12\\
$I$&13\\
$\Is$&14\\
\hline
\end{tabular}
\end{center}
\end{table}

In HUITZI it is not possible to take flats in all filters each night, so there are several block files for different sets of filters that are run in turn.

\fi